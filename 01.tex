Sherlock Holmes e a Sociedade Internacional dos Detetives Infalíveis

Carolyn Wells

\textbf{Tradução:} Francisco Araujo da Costa

\chapter{A Aventura da Mona Lisa}

Ilustrações de Reginald Bathurst Birch

Em seus aposentos na Fakir Street, os membros da Sociedade Internacional
dos Detetives Infalíveis se reuniam para uma sessão extraordinária.

--- Se algum de vocês --- o Presidente Sherlock Holmes se pronunciou
oficialmente, --- tiver alguma sugestão...

--- Meu caro Holmes --- Arsène Lupin interrompeu. --- Assim como você,
não oferecemos nem aceitamos sugestões.

--- Não --- concordou a Máquina de Pensar. --- Apenas observamos as
pistas, deduzimos a verdade e anunciamos quem é o criminoso.

--- Quais são as pistas? --- perguntou o Monsieur Lecoq perguntou para o
grupo.

Raffles olhou seriamente para o cavalheiro idoso e sorriu.

--- As pistas são a moldura atirada na escadaria dos fundos, a vaga na
parede do Louvre e os pregos sobre os quais o quadro estava pendurado.

--- A vaga na parede tem o tamanho exato da Mona Lisa? --- o Monsieur
Dupin perguntou.

--- Isso não pode ser determinado, já que o quadro não está disponível
para medirmos --- Raffles retrucou. --- Mas a Mona Lisa se foi e não há
nenhuma outra vaga inexplicada nas paredes.

--- As evidências não me parecem conclusivas --- o Monsieur Dupin
murmurou. --- A lei não determina a necessidade de um corpo de delito?

--- Essa questão não se faz presente --- Arsène Lupin interrompeu.

--- O quadro também não --- Raffles observou.

Extenuado, Sherlock Holmes passou sua mão branca sobre o cenho.

--- Ordem, por favor.

--- Cavalheiros, todos ouviram a descrição das pistas: a moldura
descartada, o espaço vazio, os pregos à vista. A partir delas, deduzo
que o ladrão tem um metro e setenta e oito centímetros de altura e pesa
setenta e três quilos. Ele tem cabelo moreno e um dente de ouro. Sua
saúde é razoável, mas um de seus primos de segundo grau sofreu de crupe
na infância.

--- Que maravilha, Holmes! Que maravilha! --- o Dr. Watson exclamou,
esfregando as mãos, estático. --- Já está praticamente atrás das grades.

--- Discordo, Holmes --- Arsène Lupin declarou. --- Na minha opinião,
está evidente que o ladrão era loiro, um tanto baixo e corpulento,
parecido com a tia-avó materna.

Holmes ficou pensativo por algum tempo.

--- Não estou enxergando, Lupin --- ele disse afinal. --- E se repassar
as pistas novamente, \emph{com cuidado}, perceberá a falácia de suas
inferências.

--- Segundo Münsterberg... --- Luther Trant começou, mas o Presidente
Holmes o interrompeu.

--- Cavalheiros --- ele disse com seu sorriso saturnino. --- Precisamos
trabalhar cientificamente neste problema. A menos que encontremos o
quadro roubado e condenemos o ladrão, não somos dignos de nossa fama
profissional. Digam-me, quanto tempo vocês acreditam que demoraremos
para alcançar nosso objetivo?

--- Em uma semana, encontro o retratinho --- o Monsiuer Dupin anunciou.
--- É preciso apenas raciocinar da seguinte maneira. Se...

--- Ora, ora --- a Máquina de Pensar disse, ranzinza. --- Quem quer
escutar os conselhos alheios? Vamos todos trabalhar independentemente
uns dos outros. Para mim, uma semana será mais do que o suficiente para
trazer tanto o quadro quanto o ladrão.

--- Uma semana, bah! --- Raffles exclamou, zombeteiro. --- Entrego a
tela desaparecida e o facínora em três dias, tenho certeza.

--- Arsène de quanto tempo você precisa para o trabalho? --- o
Presidente Holmes perguntou, mantendo seu sorriso saturnino.

--- Dois dias e o dinheiro da passagem --- Arsène Lupin respondeu. --- E
você, Holmes?

O sorriso no rosto de Sherlock Holmes se saturninizou mais um pouco.

--- Já sei onde está, só preciso ir buscar --- ele respondeu baixinho.

--- Não é justo --- Luther Trant reclamou, cortando as observações
encarecidas do Dr. Watson.

--- É perfeitamente justo --- Holmes declarou. --- Não tenho nenhuma
vantagem em relação a qualquer um de vocês. Todos ouvimos uma lista das
pistas, eu deduzi a solução do mistério. Se vocês não o fizeram, é
porque não enxergam o óbvio.

--- Sempre desconfie do óbvio --- Monsieur Dupin disse, didático.

O Presidente Holmes deu sua costumeira falta de atenção à frase e
continuou a falar.

--- Não temos mais por que conversar. Não somos um bando de consultores
amadores. Somos todos famosos, especiais e infalíveis. Vamos todos
seguir nossos caminhos, aplicar nossos diversos métodos próprios e ver
quem encontra o quadro primeiro. Em sete noites, nos reuniremos aqui
novamente. Quem trouxer a Mona Lisa receberá os parabéns do resto da
sociedade e, casualmente, a recompensa oferecida.

--- Que maravilha, Holmes! Que maravilha! --- o Dr. Watson exclamou
antes que os outros pudessem falar.

Mas não havia mais o que ser dito. Os detetives famosos estão sempre
taciturnos, silenciosos e pensativos, mas com o olhar de quem vê no
universo uma cartilha aberta.

Depois de se despedirem de diversas maneiras, os detetives foram embora
para detectar. Sherlock Holmes sacou seu violino e tocou ``Her Bright
Smile Haunts Me Still''.

Uma semana se separou lentamente do futuro e transferiu seus laços para
o passado. Mais uma vez, os aposentos da Fakir Street foram limpados e
arrumados para a reunião. Às oito, a hora marcada, ninguém apareceu.

--- Rá! --- Holmes murmurou. --- Todos falharam, mas nenhum ousou vir e
admiti-lo. Apenas eu cumpri a missão, apenas eu logrei obter a
inestimável Gioconda.

--- Que marv... --- o Dr. Watson começou, mas a porta se abriu e o
Monsieur Dupin entrou com uma tela sob o braço.

O quadro estava enrolado em um xale velho, mas pelo tamanho, e pelo
tamanho do sorriso no rosto de Dupin, até Watson deduziu que ali estava
a tela na qual Leonardo jogara suas tintas por quatro anos.

--- Ora, sim --- M. Dupin disse, distraído. --- Estou com ela. Vou
apenas esperar pelos outros, para que possa apresentar meu prêmio e
receber mais aplausos do que espero de você, Monsieur Holmes.

O sorriso de Holmes era apenas ligeiramente saturnino, mas antes que
pudesse oferecer sua resposta ácida, Lecoq apareceu com um rolo enorme,
enrolado cuidadosamente em papel. Ele tinha um sorriso alegre e
expansivo, mas quando avistou o objeto enrolado no xale, recostado
contra a parede, o francês franziu a testa.

--- O que é aquilo? --- ele bradou. --- Uma moldura dourada para a tela
que trago comigo, porventura?

Provocado além do limite por essas palavras sarcásticas, Dupin correu
até o xale e o arrancou.

--- Observem, a Mona Lisa! Recuperada! Ah, a glória!

--- Rá! --- Lecoq exclamou e, desenrolando seu rolo, também mostrou a
obra-prima de Leonardo da Vinci, original e indubitavelmente genuína.

Holmes olhou para as duas telas gêmeas, interessado.

--- Sem dúvida nenhuma, são reais --- declarou. --- Ambas. Não se pode
questionar a autenticidade de nenhuma delas. Da Vinci deve ter pintado a
senhora duas vezes.

--- Que maravilha, Holmes! Que maravilha! --- o Dr. Watson entoou.

Mas os dois franceses não estavam dispostos a aceitar a declaração de
Holmes e começaram a discutir verbosamente em seu idioma. O significado
do seu excelente francês é que ambos acreditavam ter descoberto o quadro
verdadeiro, enquanto o outro trouxera uma cópia.

A controvérsia foi interrompida pela esquisitice da Máquina de Pensar.

--- Briguem o quanto quiserem --- ele disse para os dois em sua voz
esganiçada. --- Não importa quem vencer, pois eu tenho a Mona Lisa
verdadeira em casa. Eu não arriscaria trazê-la para cá. Ambas as de
vocês são cópias, e cópias malfeitas, ainda por cima.

Foi quando apareceu Luther Trant, seguido de três mensageiros. Cada um
trazia uma Mona Lisa, que ele colocou ao lado das que já haviam chegado.

--- Uma delas é a real --- Trant anunciou. --- Não tive tempo de decidir
qual, e meu sismosfigmógrafo está estragado. Mas depois descubro. Seja
como for, é uma das três e fui eu que descobri.

No meio do burburinho causado pela declaração, Raffles apareceu com um
sorriso que brilhava de hilaridade.

--- Sucesso! --- ele gritou, e então entraram seus seguidores.

Estes eram cinco mensageiros, cuja carga total somava oito Mona Lisas:
três homens-sanduíche traziam duas Giocondas cada, enquanto duas
lavadeiras tinham um cesto com quatro.

--- Todos esses quadros foram comprovados por especialistas --- Raffles
explicou. --- Então um deles deve ser o verdadeiro.

--- Ah --- Arsène Lupin disse ao entrar. --- \emph{É mesmo?} Pois eu
tenho uma carroça na entrada com uma pilha de Mona Lisas, e para cada
uma delas obtive uma garantia assinada dos melhores especialistas.

Sherlock Holmes apenas assistia, com um sorriso que se saturninizava
cada vez mais.

--- Cavalheiros, cavalheiros --- ele disse em seus tons mais fresados.
--- \emph{Cavalheiros}, poderiam fazer a gentileza de entrar na sala ao
lado.

Ele entraram, mas delicadamente, como o rei Agague, pois o chão estava
coberto até a altura dos joelhos com Mona Lisas. Quando chegaram à sala
ao lado, foi como se entrassem em um multiscópio, pois as quatro paredes
estavam revestidas --- absolutamente \emph{revestidas} --- de Mona
Lisas. E todas, sem exceção, tinha evidências indisputáveis,
indubitáveis, impecáveis e incontroversas de serem o artigo autêntico e
genuíno.

Muito além do vexame dos detetives em saber que Holmes superara a todos,
imagine a pura alegria de poder admirar dúzias de Mona Lisas ao mesmo
tempo! Lembre dos arrepios que o arrepiaram quando estava no Louvre e
viu uma única obra-prima do grande pintor, mas então imagine aqueles
arrepios multiplicados até virarem uma febre! Foi um grande momento
psicológico.

--- São \emph{todos} autênticos? --- Monsieur Dupin sussurrou afinal
enquanto Raffles calculava seu valor coletivo para os colecionadores.

--- Todos garantidos por especialistas --- Holmes declarou, e então o
telefone tocou.

--- Sr. Holmes? --- disse o chefe de polícia.

--- Sim --- Sherlock respondeu, saturnino.

--- Sr. Holmes, gostaria de informá-lo que apreendemos a Mona Lisa. O
ladrão, que é paramaranoico, devolveu a tela e confessou o crime. Ele
está deveras arrependido e, apesar de dever ser punido, o fato de ter
confessado e então devolvido o quadro ileso com certeza será um
atenuante. Tenho certeza que o senhor ficará muito contente com a
devolução do nosso tesouro.

--- Humpf! --- Holmes disse, um pouco mais saturninamente do que de
costume, enquanto colocava o telefone no gancho. --- Quando um quadro é
devolvido tantas vezes e tão mal quanto esse, uma devolução a mais ou a
menos não importa. Agora, cavalheiros, podem começar uma imitação de
cinema e coloquem esses quadros em movimento.

\chapterspecial{Modo garantido de pegar qualquer criminoso. Rá! Rá!}{%
  Sherlock Holmes, Raffles, Arsène Lupin, M. Lecoq, Carolyn Wells e
  Outros Detetives Infalíveis Testam a Nova Descoberta Científica do
  ``Retrato Falado''}{}

A Sociedade Internacional dos Detetives Infalíveis se reunira no
escritório luxuoso da Fakir Street, desta vez para uma sessão de
indignação.

--- Um completo absurdo --- declarou o Presidente Sherlock Holmes. --- O
sistema Bertillon já é suficientemente desnecessário, mas esse
\emph{Portrait Parle} é mil vezes pior.

--- O que é isso? --- a Máquina de Pensar perguntou, ranzinza. --- O que
é um \emph{Portrait Parle}?

--- Não sabes nada de francês? --- M. Lecoq perguntou com altivez. --- É
um... um retrato que fala.

--- É um retrato falado --- Raffles interrompeu.

--- Retrato falado! --- Holmes exclamou. --- É um absurdo gritado!

--- É uma farsa berrante --- Arsène Lupin contribuiu para a opinião
geral.

--- É uma vergonha ribombante! --- Luther Trant observou, pensativo.

--- Mas o que é? --- a Máquina de Pensar reclamou. --- Alguém me conte!

--- Bem --- disse Raffles, que era sempre educado aquele senhor idoso e
rabugento. --- É uma maneira de descrever criminosos de modo que sempre
dê para reconhecê-los. É uma descrição especial de cada característica,
um registro de cada medida e um relato detalhado de todas as
peculiaridades que o suspeito possa ter.

--- Perfeitamente absurdo! --- Holmes proclamou. --- Como se não fosse
exatamente isso que deduzo a partir de pistas abstratas. As grandes
deduções que são o alicerce da minha fama! Mostre-me as pistas que eu
mesmo descrevo o \emph{Portrait Parle}!

--- Que maravilha, Holmes! Que maravilha! --- disse o Dr. Watson, mas um
tanto mecanicamente, pois estava absorto em um experimento empírico
complexo e tinha a cabeça dentro de um saco de borracha.

--- Acho que é excelente --- M. Lupin declarou. --- Se tivesse tido algo
assim para me ajudar quando era jovem, hoje seria ainda mais celebrado
do que já sou.

--- Bobagem, Lupin --- Holmes respondeu, um leve traço de saturninidade
na voz. --- Apenas um detetive defeituoso precisa desse tipo de ajuda.
Na minha opinião, o \emph{Portrait Parle} elimina todas as minhas
chances de produzir feitos espetaculares. Ele não me deixa espaço para
deduções maravilhosas.

--- E ainda por cima me deixa sem um comentário apropriado --- completou
Watson, que recuperara a cabeça.

--- A detecção não é mais como antes --- M. Lecoq reclamou. --- Até o
clima mudou, a ``neve fraca'' quase nunca mais cai no momento certo.

--- Mas não se precisa de pegadas quando se tem impressões digitais ---
Luther Trant observou.

--- Não --- a Máquina de Pensar rosnou. --- E com esse novo
\emph{Portrait Parle}, não se precisa nem de instinto de detetive.

--- É claro que não --- Holmes assentiu, amargurado. --- É o mesmo que
ver o omelete e então deduzir os ovos quebrados.

--- Que maravilha, Holmes, que maravilha --- o Dr. Watson sussurrou,
triste, temendo ser a última vez em que entoaria aquelas palavras.

Naquele momento, o telefone tocou. O Chefe de Polícia desejava falar com
a sociedade.

Por estar mais próximo do instrumento, Arsène Lupin respondeu.

--- Estamos com sorte, rapazes --- ele disse após ouvir a mensagem. ---
O Chefe quer que cacemos um criminoso foragido e está nos mandando seu
\emph{Portrait Parle}.

A informação foi recebida com uma variedade de fungadas, bufadas e
risadas, mas com a taciturnidade de um verdadeiro detetive, todos
ficaram à espera da nova ferramenta de trabalho. Um mensageiro chegou
com uma caixa, que Watson colocou sobre a mesa.

Os membros da sociedade se reuniram ao redor da mesa, pasmados e
paspalhados, enquanto o Presidente Holmes erguia a tampa.

O que encontraram parecia ser um amontoado de tralha reunida às pressas:
uma lanterna velha, uma pua, um gancho d ferro e uma machadinha. Em uma
caixa pequena, os detetives encontraram um escaravelho ou besouro
egípcio. Em outra caixa havia uma maçã e uma cenoura, enquanto uma folha
de papel-pardo enrolava uma costeleta crua. Uma caixa de confeitaria
continha uma torta tentadora.

Raffles olhava cobiçoso para a torta, mas ele era, afinal de contas, um
mero detetive diletante. Os outros, sendo o artigo genuíno, desprezavam
a ideia de comida, exceto pela Máquina de Pensar, que desejava
ardentemente mordiscar a maçã.

O Presidente Holmes cruzou os braços e lançou um olhar que era saturnino
até a ponta dos dedos.

--- O que vocês escutam nesse \emph{Portrait Parle}, cavalheiros? ---
ele perguntou.

M. Lupin enfiou as mãos nas lapelas alamaradas.

--- É um esquema perfeito --- ele anunciou em um tom oracular. ---
Observe como construímos nosso homem imediatamente. É um arqueólogo,
como nos informa o escaravelho.

--- E um açougueiro, como mostra a costeleta --- interrompeu M. Lecoq,
sempre o rival ciumento do compatriota.

--- É um confeiteiro --- sugeriu Raffles, ainda de olho na torta, que
era de merengue.

--- Um carpinteiro é mais provável --- disse Arsène Lupin. --- Observem
a pua, a machadinha e o enorme gancho de ferro.

--- E a lanterna? --- Holmes perguntou, aquilino, para variar.

--- Significa um fazendeiro --- a Máquina de Pensar gemeu, insistente.

--- Absolutamente --- Holmes disse. --- Prova que procuramos um homem
honesto.

Watson declamou algumas palavras escolhidas a dedo e depois Raffles
respondeu airosamente:

--- Mas estamos em busca de um criminoso. A lanterna significa apenas
que a questão não é sombria demais, afinal.

--- A cenoura sugere que somos asnos? --- quis saber M. Lecoq, que não
demorava para perceber sugestões implícitas.

Ninguém respondeu, no entanto, pois todos queriam desvendar o
significado do \emph{Portrait Parle}.

--- A machadinha indica algo enterrado, como fazem os índios --- Holmes
refletiu. --- E a lanterna será útil para desenterrar.

--- Não precisamos cavar à noite --- Raffles disse. --- Creio que a
costeleta e a torta indicam a hora do jantar.

--- Bem, precisaremos cavar, de um jeito ou de outro --- Holmes
persistiu.

--- É óbvio --- Lupin disse solenemente. --- Obviamente, o besouro é a
pista, como foi o Escaravelho de Ouro. É um caso de tesouro enterrado. O
Gancho, obviamente, é uma localidade, uma península ou costa rochosa.

--- E a maçã indica o Jardim do Éden, imagino --- Arsène Lupin aombou.
--- É longe demais, me recuso a ir.

--- Você é literal demais --- a Máquina de Pensar se impacientou. ---
Esses objetos são meras sugestões imaginativas. A maçã nos lembra de
Páris e Helena, de modo que, concluo, o criminoso que estamos buscando é
uma bela mulher.

--- Então vamos \emph{cherchez la femme}! --- Raffles exclamou, sempre
galante.

--- Nunca vamos fazer nada trabalhando juntos --- Holmes disse afinal.
--- Todos os detetives famosos devem se afamar sozinhos. Sigam cada um o
seu caminho, meus caros. Lembrem do \emph{Portrait Parle} e voltem
amanhã à noite com o criminoso que ele representa.

Contentes por poderem seguir seus métodos favoritos e conhecidos, os
detetives infalíveis encerraram a reunião e desapareceram.

Os membros da sociedade marcharam de volta à Fakir Street na noite
seguinte em triunfo, cada um levando o criminoso que selecionara, cada
um seguro na complacência do verdadeiro detetive de ter achado o homem
certo.

M. Lupin prendera um arqueólogo proeminente, a Máquina de Pensar
trouxera um fazendeiro abastado e arrogante e Raffles trouxe um
pasteleiro francês garboso. Cada um tinha sua presa e, quando a reunião
teve início, o Presidente Holmes se preparou para ouvir e julgar as
diversas alegações da sua própria perspectiva infalível.

O telefone tocou.

--- É o Sr. Holmes? --- o Chefe de Polícia perguntou.

--- Sim --- Holmes respondeu, asinino --- quer dizer, aquilino.

--- Bem, encontramos o criminoso que queríamos, então vocês podem
cancelar sua busca.

--- É mesmo? --- Holmes disse. --- Eu poderia solicitar que você o
trouxesse aqui para compará-lo com o \emph{Portrait Parle} que me
enviou?

--- Vou levá-lo imediatamente --- respondeu o Chefe, que era cortês e
prestativo.

Os membros da Sociedade Internacional dos Detetives Infalíveis ficaram
sentados, sérios e soturnos, até o Chefe chegar, trazendo consigo um
criminoso de aparência abjeta, que todos analisaram interessados. O
homem com certeza não era cientista e não parecia ser um fazendeiro.
Também nada indicava um carpinteiro ou um pasteleiro.

--- Infelizmente --- o Presidente Holmes entoou sarcasticamente, ---
creio que não somos totalmente fluentes no idioma do seu \emph{Portrait
Parle}.

--- Não? --- o Chefe de Polícia se surpreendeu. --- Ora, meu caro
senhor, basta olhar para este homem para ver que o \emph{Portrait Parle}
que lhe enviei é uma fotografia perfeita. Observe seus traços!

--- Ele não tem um queixo comprido de lanterna, a sobrancelha espessa do
besouro, o olhar penetrante da pua e o rosto fino como a machadinha? Ele
não tem um nariz de gancho, costeletas no cabelo cor de cenoura e boca
redonda que nem uma torta? Vocês são assim tão burros que não conseguem
entender uma descrição falada?

--- Basta, Chefe --- Holmes disse, acenando com sua longa mão branca.
--- Basta, seu \emph{Portrait Parle} é um tagarela!

\begin{enumerate}
\item
  A Aventura do Bebê Perdido
\end{enumerate}

Ilustrações de H. C. Townsend

Os membros da Sociedade Internacional dos Detetives Infalíveis estavam
reunidos em seus aposentos na Fakir Street. Era um dia muito chuvoso e
todos torciam, desesperançados, para que um caso digno de seus
intelectos individuais e coletivos lhes fosse trazido. Finalmente, como
último recurso, Arsène Lupin disse ao presidente, desesperado:

--- Olhe pela janela, Holmes! Quase sempre que olha, você vê um caso
chegando.

Dando de ombros entediado, Sherlock Holmes removeu o cachimbo do
semblante finamente esculpido e colocou-o cuidadosamente em um suporte
bordado que lhe fora presenteado por um cliente grato, um homem de tez
clara e adepto da Igreja Episcopal cujas pérolas desaparecidas ele
encontrara. Vagueando em direção à janela, ele olhou saturninamente para
a paisagem de umidade perpendicular.

--- Está tudo bem --- ele disse, melancólico. --- Lá vem ela. Uma
senhora de meia-idade, não pobre, mas um tanto parcimoniosa,
antissufragista e leitora do \emph{The Ladies' Own Ledger}. Ela perdeu
um artigo de grande valor.

Mas Holmes falou lentamente, e Watson teve tempo apenas de pronunciar a
primeira sílaba de seu clichê clássico quando a senhora foi anunciada.

--- Boa tarde, cavalheiros --- ela disse, acomodando-se na cadeira
oferecida pelo jubiloso Dr. Watson.

O Presidente Holmes olhou fixamente para ela, como se lesse e traduzisse
sua alma secreta.

Lupin, Dupin, Lecoq e Vidocq, que haviam se erguido, fizeram mesuras
francesas em ângulo reto, as mãos no coração. A Máquina de Pensar se
manteve no assento e a observou fixamente com seus olhos azuis
ranzinzas, com o queixo descansando sobre os dedos entrelaçados, que por
sua vez descansavam sobre sua bengala nodosa, que, por sua vez,
descansava, obviamente, sobre o chão. Luther Trant se mexeu na cadeira,
irrequieto, e Raffles abriu um sorriso insinuante como galanteador que
era.

--- Deduzo que está chovendo --- Holmes disse, olhando gravemente para a
visitante.

--- Isso você já sabia --- Lupin observou, olhando galicamente de
soslaio.

--- Mas ignorei --- Holmes declarou. --- Deduzi apenas pelo
guarda-chuvas e as galochas da senhora.

--- Que maravilha, Holmes! Que maravilha! --- o Dr. Watson exclamou,
emocionado até o último fio de cabelo.

--- A senhor perdeu alguma coisa, madame --- Holmes continuou,
balançando seu indicador saturnino para ela.

--- Céus, senhor! Como sabia disso? Saiu nos jornais?

--- Não. Mas tenho certeza que a senhora não está envolvida em um caso
de assassinato, e não há nenhum outro crime no mundo além do roubo.
Logo, deduzo que a senhora foi roubada. O artigo que perdeu foi...

--- Holmes! --- a Máquina de Pensar exclamou, ranzinza. --- Deixe a
senhora nos contar o que perdeu! Sobre isso, ela sabe mais do que você.

--- Não tenho certeza disso --- Holmes retrucou dubiamente, um sorriso
sério iluminando seu rosto sombrio. --- Mas continue, madame, conte-nos
o que sabe, ou acha que sabe, sobre o caso.

--- Bem, senhor, eu sou viúva.

--- Deduzi que é viúva --- Holmes interrompeu, --- assim que vi sua
aliança de casamento e o véu de crepe preto.

--- Que maravilha, Holmes! Que maravilha! --- o Dr. Watson observou, um
tanto mecanicamente.

--- Você também deduziu que ela lê \emph{The Ladies' Own Ledger} ---
Vidocq disse. --- Qual é a prova?

Languidamente, Holmes ergueu seu indicador cansado e apontou para o jabô
na garganta da senhora. Realmente, ele era feito de um pano turco,
trançado delicadamente na forma certa, trabalhado em desenhinhos com
algodão vermelho. O enfeite fora descrito na edição daquele mesmo mês,
como todos sabiam.

--- E como você sabia que ela era antissufragista? --- a Máquina de
Pensar perguntou.

--- Todos os babados em suas anáguas, que vi se chacoalhando enquanto
ela cruzava a rua.

--- E que ela era parcim...

Mas a pergunte pouco educada de Lecoq foi interrompida por Raffles, que
colocou as duas mãos sobre a boca do colega.

--- Mas que disparate! --- Holmes exclamou. --- Se ela podia usar essas
roupas de baixo extravagantes, ela poderia se dar ao luxo vir de táxi,
mas não viera, ela estava... estava... caminhando por motivo de saúde
--- Holmes concluiu, pois a senhora olhava fixamente para o presidente.

--- Meu nome é Plummer --- ela começou. --- Sra. Ezra J. Plummer, mas
suponho, senhor, que saberia disso também se eu não tivesse lhe contado.

--- Obviamente --- Holmes respondeu, descuidado. --- Continue.

--- Eu moro sozinha desde que Ezra morreu, vai fazer dezenove anos em
junho, e cuido da minha casa como sempre cuidei. Não sou dessas de mudar
meus móveis cada vez que a moda decide que está na hora. As cadeiras de
veludo na minha sala de estar ainda estão tão boas quanto no dia em que
as compramos; duas são vermelhas, três são verdes e o sofá é vermelho.
Elas têm armações de ébano, com realces em dourado, e não se acha em
lugar nenhum um conjunto mais bem arrumado do que o meu.

--- Que mobiliário charmoso --- Raffles disse, polido. --- Uma bela
ideia essa a de alternar vermelho e verde. E a senhora perdeu essas
cadeiras, Madame?

--- Não, senhor, ladrões não roubam cadeiras. O que eu perdi foi uma
obra de arte, a decoração principal da minha sala, meu bem mais valioso.
Um tesouro! --- e com isso a Sra. Plummer perdeu o controle e pôs-se a
chorar.

Os quatro cavalheiros franceses, tendo personalidades solidárias e
emotivas, também choraram. A Máquina de Pensar se remexeu na cadeira,
desconfortável.

O Presidente Holmes olhou pela janela com os braços cruzados
placidamente.

--- Uma obra de arte! --- sibilou. --- Rá, um caso paralelo ao da Mona
Lisa! O que foi, madame? Um quadro, uma estátua?

--- Ah, como você é esperto! --- ela exclamou. --- Está chegando perto.
Tente de novo!

--- Uma estatueta, uma antiguidade, uma raridade, um bronze? --- os
detetives ansiosos sugeriram um depois do outro.

--- Não! --- a Sra. Plummer exclamou. --- Vocês nunca vão adivinhar! Foi
um Rogers Group.

--- Rogers Group! O que é isso? --- Lecoq perguntou, pois a fama do
Grande Agrupador jamais penetrara sua terra natal inculta.

--- Ah, senhor --- a Sra. Plummer exclamou. --- Um dos modelos mais
primorosos! Era ``Pesando o Bebê'' e, ah, se ao menos o senhor pudesse
ver a cena, o médico idoso erguendo os óculos, a enfermeira com as mãos
firmes, a criancinha... ai, a criancinha! Sumiram!

Mais uma vez ela se abateu e começou a chorar, como fazem as mulheres
quando o assunto são bebês. Os quatro cavalheiros franceses, solidários,
seguiram seu exemplo copiosamente.

--- Ah, um caso de sequestro! --- Luther Trant exclamou.

--- Quanto pesava o bebê? --- a Máquina de Pensar indagou, tenso.

--- Continue, madame --- o Presidente Holmes interrompeu. --- Conte os
detalhes do roubo.

--- Bem, meus senhores, foi assim. Esta tarde saí para o encontro da
Sociedade de Costura e, obviamente, tranquei a casa como sempre faço. O
Rogers Group estava na sala de estar, sobre uma mesa com tampo de
mármore, descansando sobre um lenço de pelúcia vermelho-granada.
Senhores, todas as janelas da sala de estar estavam protegidas por
pega-ladrões e a porta da frente estava chaveada. Em suma, todas as
portas e janelas estavam absolutamente trancadas.

--- Em outras palavras, a sala de estar estava hermeticamente selada!
--- Luther Trant declarou.

--- Rá! Hermeticamente selada! --- exclamou Rouletabille. --- É tudo que
um caso precisa para ficar interessante! Conte mais!

--- Saí às duas horas --- a Sra. Plummer continuou, dramática. --- Saí
às duas, mas às quatro, quando voltei, o Rogers Group havia sumido! Sem
deixar vestígios, senhores. Desaparecidos, o bebê e o doutor, a balança
e a enfermeira... desaparecidos!

--- Desaparecidos! Desaparecidos! --- Dupin ecoou, torcendo as mãos. O
detetive muitas vezes era dominado por um excesso de solidariedade,
assim como acontecia com os outros cavalheiros franceses.

--- E a casa hermeticamente selada! --- ponderou Rouletabille,
exultante. --- Não existe problema mais delicioso! Vocês lembram que em
``O Quarto Amarelo'', eu...

--- Havia alguma pista? --- o Presidente Holmes perguntou, cortando
propositalmente as reminiscências do francesinho.

--- Não sei, senhor --- ela respondeu. --- Ouvi falar que não se toca no
cadáver até o legista chegar, então achei que o mesmo valeria em caso de
roubo. Então tranquei a casa de novo e vim correndo.

--- Muito correta --- Holmes aprovou, saturnino. --- Vou para lá
imediatamente. Venha, Watson.

Apesar de aparentemente ignorados, os outros recolheram seus chapéus e
saíram voando pela porta, ansiosos por serem os primeiros na cena do
crime, como apenas verdadeiros detetives conseguem ser.

A chuva havia parado, então o grupo foi caminhando a passos largos pelo
pavimento ainda úmido. Finalmente, a Sra. Plummer destrancou solenemente
sua porta de entrada e conduziu os dez homens até o local.

--- A sala! Qual é a sala certa? --- Rouletabille perguntou, rouco, pois
aquele era justamente o tipo de caso que deleitava sua alma.

--- Aqui! --- e a Sra. Plummer abriu a porta com um gesto dramático.

Como anunciado, a janela saliente na qual o Rogers Group ficara exposto
orgulhosamente por dezenove anos estava vazio. Desaparecida, uma obra de
arte inestimável! Desaparecidos, o médico gentil, a enfermeira orgulhosa
e o bebê rechonchudo!

--- Rá! Pegadas! --- o Presidente Holmes murmurou.

Em um instante, Holmes estava de joelhos com a lupa, o compasso e a
régua T. Mas a lupa era desnecessária, pois as pegadas eram de tamanho
considerável. Com todo o cuidado, Holmes colocou um diagrama de escala
e, com a ajuda de alguns dos outros, recortou uma matriz de papel
exatamente igual às pegadas, para a seguir entregar uma cópia a cada um
dos membros do clube. Com essa pista, eles rastreariam o criminoso.

--- E vamos conseguir! --- disse Vidocq, confiante.

--- Ora, essas pegadas são minhas --- a Sra. Plummer disse, como se as
palavras lhe fossem forçadas pela consciência. --- Quando cheguei, a
lama ainda est...

--- Por que não nos disse isso logo no princípio? --- Trant quis saber.

--- Vejam vem, eu estava com meus sapatos velhos, e eles sempre foram
grandes demais para mim mesmo.

--- Que belo exemplo do eterno feminino! --- Trant comentou. --- Um
instante! O facínora deve ter deixado impressões digitais. Vou
fotografar este cobertor de chenille de pelúcia e essas cadeiras
apeluciadas, com sorte encontro uma impressão do polegar.

Mas os próximos momentos revelaram resultados surpreendentes. Dúzias de
impressões digitais foram identificadas nas superfícies empoeiradas das
peanhas e do consolo da lareira. Raffles encontrou um tufo de penas, sem
dúvida nenhuma caído da boá de uma senhora. A Máquina de Pensar
encontrou um lenço com o monograma ``G'', Dupin encontrou uma carta
velha, Vidocq um estojo de óculos e Lecoq uma luva. Raffles achou também
um barrete cinza e Holmes recolheu uma lista de compras.

--- Cavalheiros --- o Presidente convocou. --- Todos têm suas pistas
individuais. Sigam pelos seus caminhos, façam suas deduções e compareçam
aos nossos aposentos amanhã, quando vou mostrar a todos vocês quem foi o
ladrão.

Os Detetives Infalíveis foram cada um para um lado, secretamente
enfurecidos com a arrogância de Holmes.

No dia seguinte, às três horas da tarde, todos marcharam de volta até os
aposentos da associação, cada um trazendo consigo uma moradora da
cidade.

--- Rá! --- Holmes exclamou. --- O vilão parecer ser plural.

--- E feminino --- a Máquina de Pensar completou, olhando de soslaio
para a senhora rechonchuda que capturara.

--- Primeiro temos que tirar fotos de todas elas --- Holmes declarou.

--- Já esperávamos --- disse a Sra. Green, identificada pelo ``G'' de
seu lenço e que atuava como porta-voz do grupo. --- Estamos com nossos
melhores vestidos de propósito. Vamos tirar fotos individuais ou em
grupo?

As senhoras ficaram agitadas, na expectativa agradável de serem
fotografadas. Após completado o retrato, os detetives questionaram suas
cativas, todas as quais haviam identificado facilmente pelas diversas
pistas deixadas para trás. Todas declararam que haviam estado presente
na sala de estar da Sra. Plummer entre as duas e as quatro horas da
tarde passada.

--- Então vocês confessam que furtaram o Rogers Group da Sra. Plummer?
--- Holmes perguntou.

--- Confessamos! --- as senhoras entoaram em uníssono.

--- Admitem que o levaram com intenções criminosas, em outras palavras,
que o roubaram?

--- Sim --- as senhoras declararam unanimemente. --- E você não pode nos
prender por isso, pois podemos provar que estávamos no nosso direito.

--- Provem --- o Presidente Holmes disse.

--- Eu sou a presidente --- a Sra. Green começou. --- Estas senhoras são
membros de nossa Associação de Melhoramentos da Vila. No interesse da
nossa obra, muitas vezes somos forçadas a remover...

--- Ah, sim --- Holmes exclamou. --- Entendo muito bem. Sim, sim, muito
bem! Nem mais uma palavra, eu lhes imploro, minha cara! Tudo está
explicado. As senhoras estão escusadas e a Sra. Plummer não tem como
proceder, absolutamente, não tem. Muito boa tarde, senhoras.

--- Ah, sim, mas peço que se detenham por mais um momento --- disse
Rouletabille, arregalando seus olhos ansiosos, ainda cheio de interesse
pelo caso. --- Por favor, posso pedir a solução da única pergunta que me
interessou neste caso? Como vocês entraram na casa selada
hermeticamente?

A Sra. Green olhou com pena para o detetive.

--- Meu senhor, eu peguei a chave debaixo do capacho e depois coloquei
de volta no lugar.

\begin{enumerate}
\item
  A Aventura do Varal
\end{enumerate}

Ilustrações de Frederic Dorr Steele

Os membros da Sociedade dos Detetives Infalíveis estavam sentados, sendo
socialmente infalíveis em seus aposentos na Fakir Street, quando o
Presidente Holmes apareceu. Ele estava muito mais saturnino do que de
costume e os outros deduziram imediatamente que algo acontecera.

--- E foi o seguinte --- Holmes disse, percebendo que eles haviam
percebido. --- Foi oferecida uma recompensa pela solução de um grande
mistério, tão grande, caros colegas, que temo que nenhum de vocês será
capaz de resolvê-lo, ou sequer de me ajudar no trabalho maravilhoso que
produzirei para desvendá-lo.

--- Humpf! --- grunhiu a Máquina de Pensar, cravando seus olhos azuis
acerados no falante.

--- O colega fala por todos nós --- Raffles disse com seu sorriso
sedutor. --- Abra o jogo, Holmes. O que é que há?

--- Explicar um ocorrido deveras misterioso que sucedeu no Lado Leste da
cidade.

Apesar de ser um homem alto e magro, Holmes falava curto e grosso, pois
estava zangado com a atitude desatenta de seus colegas reunidos. Mas ele
ainda tinha seu Watson, é claro, então aturava a indiferença do resto do
mundo frio ao seu redor.

--- Os ocorridos no Lado Leste não são todos misteriosos? --- Arsène
Lupin perguntou com um olhar aristocrático.

Holmes passou o cenho cansado por sob a mão.

--- O Inspetor Spyer percorria a Estrada Elevada, uma das avenidas de
número menor da cidade, quando, passando pelo distrito dos cortiços, viu
um varal estendido de uma janela de andar alto até outro sobre um pátio.

--- Era uma segunda-feira? --- perguntou a Máquina de Pensar, que por um
instante pensou que era uma máquina de lavar.

--- Isso não importa. Aproximadamente no meio da corda estava
suspensa...

--- Por prendedores? --- perguntaram dois ou três dos Infalíveis ao
mesmo tempo.

--- Estava suspensa uma linda mulher.

--- Enforcada?

--- Não. \emph{Escutem!} Ela estava pendurada pelas mãos e evidentemente
tentava cruzar o varal de um apartamento até o outro. Pelo rosto exausto
e agoniado, o inspetor temeu que ela não conseguiria se sustentar por
muito mais. Ele saltou do seu assento e correu para ajudá-la, mas o trem
já havia partido e ele não conseguiu descer a tempo.

--- O que ela estava fazendo?

--- Ela caiu?

--- Como ela era?

Essas e várias outras perguntas sem sentido caíram dos lábios dos
grandes detetives.

--- Fiquem quietos e contarei todos os fatos conhecidos. Ela era um
membro da alta sociedade, isso está claro, pois vestia um vestido de
baile de \emph{chiffon}, uma daquelas peças com o topo caído. Ela vestia
joias elegantes e sapatos delicados, com fivelas encrustadas de joias. O
cabelo, livre de suas amarras, pendia em cachos pesados até as costas.

--- Que extraordinário! Qual é o significado disso tudo? --- perguntou
Monsieur Dupin, que era sempre claro e direto.

--- Ainda não sei --- Holmes respondeu honestamente. --- Tenho estudado
a questão há apenas alguns poucos meses. Mas vou descobrir, mesmo que
tenha que demolir todos os cortiços da quadra. Deve haver uma pista em
\emph{algum} lugar.

--- Que maravilha, Holmes! Que maravilha! --- disse um fonógrafo no
canto da sala, que Watson havia instalado, pois ele precisara sair.

--- A polícia pediu que investiguemos o caso e ofereceu uma recompensa
pela solução. Descubram quem era a dama, o que estava fazendo e o
porquê.

--- Há alguma pista? --- perguntou Monsieur Vidocq.

--- Alguma pegada? --- acrescentou Monsieur Lecoq simultaneamente.

--- Uma pegada, nenhuma outra pista.

--- Onde está a pegada?

--- No solo, abaixo de onde a dama estava dependurada.

--- Mas você disse que a corda ficava suspensa bem no alto.

--- Mais de trinta metros de altura.

--- E ela desceu e deixou uma única pegada. Estranho! Muito estranho!
--- e a Máquina de Pensar balançou sua cabeça idosa e amarelada.

--- Ela não fez nada disso --- Holmes disse, petulante. --- Se vocês se
dessem ao trabalho de escutar, quem sabe ouvissem alguma coisa. Os
moradores dos cortiços foram questionados, mas nenhum deles estava em
casa na hora do ocorrido. Havia uma parada na rua ao lado e todos haviam
saído para assistir.

--- E havia caído uma neve fraca na noite anterior? --- Lecoq perguntou,
ansioso.

--- É óbvio que sim --- Holmes respondeu. --- Se não, como saberíamos
alguma coisa? Bem, a dama deixou cair seu sapato e, apesar do item não
ter sido encontrado, tendo sido anexado pelos primeiros moradores do
cortiço a chegarem em casa, eu tive a oportunidade de estudar a pegada.
O sapato era tamanho 33, pequeno demais para ela.

--- Como você sabe?

--- As mulheres sempre usam sapatos pequenos demais para elas.

--- Então como ela deixou cair? --- Raffles retrucou, triunfante.

Holmes lançou um olhar de pena.

--- Ela o atirou porque ele estava apertado demais. As mulheres sempre
tiram os sapatos quando jogam \emph{bridge}, ou no camarote da ópera, ou
durante o almoço.

--- E sempre quando estão cruzando um varal? --- Lupin perguntou com o
máximo de sarcasmo.

--- Naturalmente --- Holmes respondeu, franzindo a testa taciturnamente.
--- A pegada denota claramente uma dama rica e conhecedora da moda, de
estatura relativamente baixa, pensado cerca de setenta quilos. Ela era
de natureza animada...

--- Animação suspensa --- Luther Trant brincou.

--- Como na Caixa de Dâmocles, ou seja lá quem for --- completou
Sprague, o Científico.

Mas Holmes desaprovava dessa leveza de espírito.

--- Precisamos descobrir o que isso significa --- ele disse em seu tom
mais melancólico. --- Já tenho o traçado da pegada.

--- Gostaria de saber se o sismosfigmógrafo vai funcionar nele --- Trant
refletiu.

--- Eu sou o Príncipe das Pegadas --- Lecoq declarou, pomposo. ---
\emph{Eu} vou solucionar o mistério.

--- Façam o melhor que puderem, todos vocês --- disse o ilustre
presidente. --- Temo que não terão muito sucesso, pois essas questões
são ininteligíveis para os ininteligentes. Mas estudem o caso e voltem
aqui em uma semana, com suas respostas datilografadas, sem erros de
ortografia, em uma única página.

Os Detetives Infalíveis partiram, todos demonstrando um comportamento
caracterizado por uma alegria sanguínea que em nada foi capaz de
impressionar seu presidente, acostumado como era a impressões
sanguinárias.

Eles dedicaram os sete dias reservados ao estudo do problema, e boa
parte das sete noites também, pois todos queriam mergulhar naquele
segredo desconcertante à luz de vela e à luz do sol, como a Sra.
Browning expressou poeticamente.

E quando a semana passou, os Infalíveis se reuniram mais uma vez no
santuário da Fakir Street, um sorriso arrogante estampado em cada um dos
rostos, o semblante de alguém que saíra ao encalço de uma presa que
agora estava devidamente presa e estava preparado para receber sua justa
recompensa.

--- E agora, --- o Presidente Holmes anunciou --- como nada pode ser
ocultado dos Detetives Infalíveis, presumo que todos descobrimos por que
a dama estava pendurada do varal acima do abismo profundo e perigoso que
é o pátio de um cortiço.

--- Descobrimos --- responderam seus colegas, em diversos tons de
orgulho, arrogância e falsa modéstia.

--- Não consigo imaginar --- o presidente continuou com sua voz de
gavião --- que algum de vocês encontrou a verdadeira solução desse
mistério, nem por acaso, mas estou disposto a escutar suas tentativas
amadoras.

--- Como membro mais velho da nossa organizações, contarei minha solução
primeiro --- Vidocq disse, calmo. --- Não consegui encontrar a dama, mas
estou convencido que ela era meramente uma trapezista ou equilibrista de
extrema habilidade, praticando um novo truque para impressionar seu
público em Coney Island.

--- Bobagem! --- Holmes exclamou. --- Nesse caso, a dama vestiria uma
malha. Fomos informados que ela usava um vestido de baile completo, da
última moda.

Arsène Lupin foi o próximo a falar.

--- É fácil demais --- ele disse, entediado. --- Ela era uma datilógrafa
ou estenógrafa que, incomodada pelas atenções de seu empregador, estava
tentando escapar do monstro.

--- Mais uma vez, chamo sua atenção para as vestes que usava --- Holmes
disse, com um ar de intolerância em seu rosto finamente esculpido.

--- Sem problema --- Lupin devolveu, relaxado. --- Essas meninas se
vestem de qualquer jeito! Já vi elas por aí, não veem nada de errado em
usar vestidos de gala para trabalhar.

--- Humpf! --- disse a Máquina de Pensar, e todos os outros concordaram.

--- Próximo --- Holmes disse, inflexível.

--- Eu sou o próximo --- Lecoq começou. --- Sugiro que a dama havia
escapado de um hospício para lunáticos das redondezas. Ela tinha a
ilusão de que era um casaco velho e que as traças lhe haviam atacado.
Assim, obviamente, ela se pendurou no varal. Essa teoria da insanidade
também explica o fato do cabelo da dama estar solto. Ela estava como
\emph{Ofélia}, ora.

--- Teria sido mais fácil engolir algumas boas bolas de naftalina ---
Holmes disse, olhando para Lecoq em um silêncio tempestuoso. --- O Sr.
Gryce é um deduzidor experiente. Qual foi a \emph{sua} conclusão?

O Sr. Gryce grudou os olhos na ponta da bota direita, seguindo seu
famoso hábito.

--- Na minha opinião, ela estava fazendo uma visita aos bairros
miseráveis. Sabem como é, as senhoras da alta sociedade adoram essas
coisas. E creio que ela pertencia ao Culto da Melhoria dos Varais, no
qual atua como testadora. Sua função é atravessá-los com as mãos e, se
eles conseguem sustentar seu peso, são aprovados pelo censor.

--- E se não conseguem?

--- Evidentemente, esse transtorno não havia ocorrido na época do nosso
problema, então não pode ser considerado.

--- Creio que Gryce está certo sobre a visita --- Luther Trant observou.
--- Mas o motivo para a dama estar pendurada do varal é a necessidade
imperativa que sentia de se arejar totalmente após conhecer os cortiços.
Esses locais exalam um odor específico que, se me dão a licença, exige o
uso de ozônio em grandes quantidades.

--- Vocês são materialistas demais --- disse a Máquina de Pensar, com
uma expressão distante nos olhos azuis pálidos. --- A dama é discípula
do Novo Pensamento. Ela precisa se colocar em silêncio, ou se
concentrar, ou seja lá como chamam. E elas sempre escolhem os lugares
mais estranhos para essas sessões de reflexão. Elas precisam de solidão
e, pelo que entendo, o varal não estava tumultuado, estava?

Rouletabille respondeu com uma gargalhada.

--- Você está enganado, Máqui. O problema da moça é só que ela queria
emagrecer. Li muito sobre isso nas revistas femininas, elas todas querem
emagrecer. Vivem fazendo exercícios malucos e essa história de cruzar
varais com as mãos é só mais um. Aposto que ela perdeu uns dez daqueles
quilinhos que o Sherly disse que ela tinha.

--- Ora, puxa e francamente! --- Raffles observou com seu linguajar da
alta sociedade. --- Vocês não me enganam. A espertinha estava inventando
uma nova dança que vai ser a sensação do inverno. Preparem-se para as
manchetes de domingo: ``Dança Incrível! A Valsa do Varal é a Nova
Coqueluche!'' Era \emph{isso} que estava acontecendo. O que vocês acham,
hein?

--- Vá dar uma volta, Raffles --- Holmes disse, caridoso. --- Você ainda
está dormindo. Sprague, o Científico, você às vezes propõe uma teoria
mais intricada. O que diz desta vez?

--- Não precisei de ciência --- Sprague disse, distraído. --- Assim que
ouvi que ela estava com o cabelo solto, deduzi a conclusão correta. Ela
havia lavado o cabelo e estava secando-o. Minha irmã sempre coloca a
cabeça para fora da claraboia, mas o plano dessa dama foi, na minha
opinião, mais bem-sucedido.

Agora que todos haviam anunciado suas teorias, o Presidente Holmes se
ergueu para oferecer o benefício inestimável da sua própria opinião.

--- Suas ideias têm algum mérito --- ele admitiu. --- Contudo, estão
ignorando o elemento do eterno feminino no problema. Assim que contar a
solução verdadeira, todos vão se perguntar como não a enxergaram. A dama
pensou ter ouvido um camundongo, então saiu correndo pela janela,
preferindo arriscar sua vida no varal do que permanecer nos mesmos
aposentos que um camundongo. Foi tudo muito simples. Ela estava se
penteando e jogou a cabeça para a frente para torcê-lo, como elas sempre
o fazem, e então espiou o camundongo sentado no cantinho.

--- Que maravilha, Holmes! Que maravilha! --- o Dr. Watson exclamou,
voltando de sua pequena missão.

Enquanto ainda ponderavam a sabedoria suprema de Holmes, o telefone
tocou.

--- Estás aí? --- disse o Presidente Holmes, sempre impecavelmente
britânico.

--- Sim, sim --- respondeu a voz impaciente do chefe de polícia. ---
Pode suspender a busca e chamar de volta os detetives. Descobrimos quem
era a mulher cruzando o varal e por que ela fez isso.

--- Não imagino que saiba de fato --- Holmes disse para o transmissor.
--- Mas me conte o que acha que sabe.

--- Argh! É claro que sei! Não passou de um daqueles benditos golpes
publicitários para o cinema!

--- É mesmo? E por que a senhora tirou o sapato?

--- Argh! Era parte da história idiota. O nome dela é Flossy Flicker, da
Flim-Flam Film Company, e está filmando um suspense de seis rolos,
\emph{Com a Corda Toda}.''

--- Ah --- Holmes respondeu suavemente. --- Meus parabéns à Srta.
Flicker pelo excelente trabalho.

--- Que maravilha, Holmes! Que maravilha! --- o Dr. Watson disse.

\chapter{Cherchez la Femme}

%Ilustrações de Rea Irvin

A Sociedade dos Detetives Infalíveis esperava algo para engolir. Havia
muito tempo que os membros não enfrentavam um caso que pudessem
realmente celebrar e seus intelectos enferrujavam com o desuso.

--- Olhe pela janela, Holmes! --- Watson disse, petulante, para o
presidente saturnino da Sociedade. --- Quase sempre que olha, você vê
alguém se aproximando e acaba sendo um caso.

Deixando de lado sua agulha hipodérmica, Holmes deu de ombros e se
dirigiu à janela, onde ficou observando a cena da rua, tenso e
melancólico.

--- Alguém está entrando --- ele disse lentamente. --- Pode ser que seja
um caso. Se não me engano, é a pegada dele que escuto nas escadas.

Antes de Holmes terminar de falar, alguém bateu à porta. Um jovem rapaz
de vinte e seis verões e meio no outono seguinte adentrou a sala e se
atirou, desanimado, na exata cadeira para a qual Holmes lhe indicava.

--- Como você entrou no elevado da Nona Avenida na altura da Rua 93 e
saiu na Sexta Avenida com a 28, você não podia simplesmente deixá-lo no
conserto, ora --- Holmes observou, solidário.

--- Não --- o jovem respondeu, desalentado. --- E o diabos é que...

--- Sim, eu sei, esses relógios de escapamento cilíndrico não. Mas que
problema o traz a nós?

O jovem olhou com o ar pasmo que mais cedo ou mais tarde sempre aparece
nos olhos dos clientes de Holmes.

--- Mas ora, como você sabia onde eu entrei e onde saí? Como sabia que
meu relógio se atrasara? Como...?

--- Eu sei mais do que isso, Sr. Elmer Ensign. Mas por que está atrás
dela nas cozinhas alheias?

O visitante ficou de queixo caído. Era um queixo quadrado, jovem,
barbeado, mas caiu ainda assim com a surpresa intensa registrada por seu
proprietário.

--- Elementar, meu caro, elementar. Mas o tempo está se esvaindo. Não
seria melhor apresentar seu caso? Estamos aqui todos reunidos, nosso
bandinho de Detetives Infalíveis, e se alguém pode resolver seu
mistério, somos nós.

--- Bem, cavalheiros, é um caso de sequestro.

Vários ``ahs'' escaparam dos semblantes esfíngicos dos detetives. Dupin
e Lupin esfregaram as mãos como legítimos franceses, enquanto Lecoq e
Vidocq deram de ombros, também ao modo de sua cidade natal.

A Máquina de Pensar piscou seus olhos azuis pálidos e idosos e o Sr.
Gryce concentrou seu olhar fixamente em uma conchinha no porta-bibelô.

--- Sim --- o Sr. Ensign continuou. --- Titia...

--- Gracie Golightly --- Holmes observou com destreza e cortesia.

--- Sim --- Ensign respondeu bruscamente. --- Se conhece a história tão
bem, por que não conta você mesmo?

--- Continue --- a Máquina de Pensar disse, irritado. --- Dois e dois
são quatro, não agora, mas todas as vezes. Continue de uma vez!

--- Bem, cavalheiros, é o seguinte: Apesar de ser minha tia, eu não a
via a anos até a noite passada. Ela chegou em casa às nove e disse que
estava decidida a alterar o testamento. Ela ia deixar sua fortuna,
sabem, para...

--- Ela é Golightly, a dançarina? --- Lupin perguntou com interesse
renovado.

--- Era. Ela deixou os palcos. Ela... hã... bem, ela...

--- É mais jovem do que já fora --- Holmes disse, saturnino.

--- Exatamente, muito bem colocado --- e o Sr. Ensign riu, nervoso. ---
Bem, ela tem uma causa filantrópica favorita e deixou todo o seu
dinheiro para ela, mas depois mudou de ideia e veio nos contar. Ficamos
um tempo conversando sobre o assunto, depois chegou a hora de irmos para
a cama. Hoje de manhã, Titia sai para dar sua caminhada matinal, o rubor
da juventude de volta ao rosto, uns passos ligeiros sob o sol, esse tipo
de coisa. Mas não voltou. Saímos em busca, corremos o dia inteiro até o
meio-dia. Daí recebemos uma mensagem misteriosa, datilografada. Olhem
só!

O papel que ele mostrou lia:

\begin{quote}
Temos gracy Golitely. Vamos fica co ela até resseber o resgate não
inporta o que ela fizer. Mande cinqenta mil dóleres como mandamos ou não
vae mais ver sua Tia dinovo!!!
\end{quote}

Séq Uestro.

--- Qual era a sua causa filantrópica favorita? --- perguntou Craig
Kennedy, as sobrancelhas se encontrando acima do nariz enquanto
resmungava a pergunta.

--- Ah, sim, era uma causa muito nobre --- disse o jovem Ensign. ---
Fornecer copeiros para copas descopeiradas. Vocês sabem que qualquer
casebre tem sua própria copa, mas os donos nem sonham em colocar um
copeiro dentro dela. Minha Tia Gracie decidiu que era uma vergonha ver
tantas copas sendo desperdiçadas e dedicou sua vida a alojar copeiros
dentro delas e deixou toda a sua fortuna para a causa. Mas de uma hora
para a outra ela se desgostou da ideia e decidiu deixar a grana para mim
e Minna. Minna é minha mulher. Titia veio lá em casa resolver a questão
ontem à noite e agora ela... ela foi sequestrada. Obviamente, a
Associação dos Copeiros está por trás disso. Os copeiros contrataram o
tal Séq Uestro para cometer o crime, mas são eles que estão financiando
o esquema. Vocês podem resgatá-la, por favor?

--- Vamos! --- os detetives anglófonos proclamaram em coro.

--- \emph{Cherchez la femme!} --- exclamaram os franceses.

--- Muito bem, \emph{chassez} e \emph{cherchez} --- Holmes disse, e isso
é mais difícil de dizer do que parece.

--- Há alguma pista? --- Rouletabille perguntou, balançando sua cabeça
redonda em um círculo.

--- Aqui está um retrato de Titia --- o Sr. Ensign disse, retirando uma
fotografia do bolso.

Parecia... bem, você sabe como é a fotografia de uma dançarina
profissional. Era um estudo em movimento emocional. Todos a estudaram
por um longo instante.

--- Sua tia? --- Holmes disse, duvidoso.

--- Bem, tia-avó --- o jovem Ensign corrigiu. --- Ela...

Mas ninguém o ouviu, estavam todos fotografando as impressões digitais
no cartão.

--- Não suponho que o senhor tenha uma gota do sangue dela, tem? ---
Craig Kennedy disse, abstraído. --- Não? Que pena. Gostaria de
experimentar minha sismosfigmagonça nele. Seria...

--- Mas \emph{que} pérolas! --- comentou Arsène Lupin, indicando a
joalheria montada sobre a garganta elongada de Gracie Golightly.

--- Eram mesmo --- o sobrinho suspirou. --- Mas ela as vendeu para
ajudar aqueles copeiros de uma figa! Eu gostaria de trazê-la de volta
antes que perca todo o resto dos penduricalhos. Vocês sabem como são
essa gente de teatro!

--- Alguma outra pista? --- Holmes saturninizou.

--- Sim, uma abotoadura quebrada e farrapos de um material de lã negra,
recolhidos onde o corpo não foi encontrado. Ah, senhores, por favor,
vocês acham que encontrarão minha querida titia?

Havia algo de patético na tristeza daquele jovem.

--- Sim, é claro --- Holmes respondeu no mesmo instante. --- Essas
pistas resolvem a situação perfeitamente. Vejo que o sequestrador tinha
1,75 m de altura, usa meias tamanho dez e meio, reparte o cabelo no lado
e teve sarampo na infância. O vestido de sua tia é um pouco antiquado
--- ele completou friamente para o cliente.

--- Sim --- Ensign concordou. --- Mas é melhor do que nada.

--- Talvez --- Lecoq disse. --- Começamos agora, chefe?

--- Sim --- Holmes respondeu com um leve tom de saturninidade. --- Vão
em frente e \emph{cherchez} essa \emph{femme}.

--- Caiu uma neve fraca? --- Dupin perguntou, ansioso.

--- Sim, é claro, as pegadas estão à espera. Vão!

Eles foram.

Holmes roçou as pontas dos dedos brancos da têmpora esquerda até a
direita, então alcançou seu violino e começou a tocar ``When We Were
Twenty-one''.

--- Grace Golightly! --- ele disse, nostálgico. --- Foi em ``The Black
Crook'' ou...

--- Mas que coisa --- Ensign, que ainda não havia saído, interrompeu.
--- Como você sabia que meu relógio estava parado?

--- Quando entrou, você olhou para o relógio de pulso e depois para o
meu relógio de parede. Como o seu estava dez horas atrasado, deduzi que
ele estava parado. Esses brinquedos com escapamentos cilíndricos não...

--- Sim, eu sei. Mas e quanto a entrar na Rua 93 e sair na 28...

--- Elementar, absolutamente primário. Acabaram de pintar um hidrante de
vermelho na Rua 93 e uma caixa de correio de verde na 28. Você absorveu
uma pincelada de cada no casaco enquanto passou por eles.

--- Rá! Você tira toda a graça! Mas como sabia que eu correra atrás de
Titia nas cozinhas?

--- Estamos em uma manhã de segunda-feira e senti o odor de espuma de
sabão. Imaginei que estava correndo atrás de copas sem copeiros e que as
senhoras estavam lavando suas próprias roupas.

--- Exato! Agora tenho que voltar para o escritório. Quando vocês vão
ter Tia Gracie de volta?

--- Logo, creio. Não se preocupe. Vou telefonar quando pegarmos esse Séq
Uestro com a mão na cumbuca. \emph{Au revoir}, meu caro, e diga a sua
esposa para não usar sapatos pequenos demais para os pés dela.

--- Deus me abençoe! Como você sabia que ela faz isso?

--- As mulheres sempre o fazem. Até.

Ensign partiu, e o jovem suplente que ocupava o lugar de Watson disse
``que maravilha, Holmes, que maravilha'', mas com tão pouco entusiasmo
que precisou ensaiar a fala duas ou três vezes.

Saltemos agora para o momento em que os detetives voltaram de sua
missão. Todos, sem exceção, anunciaram o mais absoluto fracasso.

Sherlock Holmes esteva enojado.

--- Mas que belo bando de detetives infalíveis --- ele saturninizou. ---
Tenho vontade de renunciar ao cargo de presidente desta sociedade.

Os outros pareceram esperançosos, mas Holmes era um homem de muitas
ideias e nenhum confiou muito na sugestão.

--- Saí e acampei na copa de uma quitinete --- Vidocq explicou. ---
Obviamente, ela apareceria por lá mais cedo ou mais tarde.

--- Não se ela foi sequestrada --- Dupin protestou. --- De minha parte,
me dirigi imediatamente à Associação dos Copeiros para lhes perguntar
sobre o caso, mas eram todos uns esnobes e não consegui marcar uma
audiência.

--- Não estou interessado em ouvir suas desculpas --- o Presidente
Holmes disse, parecendo distraído e desalentado. --- Se achasse que
vocês dariam uma bola fora com um caso tão simples, teria ido eu mesmo.

--- Encontrei uma mulher que provavelmente era a Sra. Golightly ---
Dupin afirmou. --- Ela disse que não era, mas você sabe como são as
mulheres, não conseguem dizer a verdade nem quando tentam, então ouso
dizer que era ela.

--- E por que não a trouxe?

--- Ela se recusou. Você sabe como são as mulheres, se quer que façam
alguma coisa, é impossível obrigá-las.

--- Não se pode pegar uma mulher --- Arsène Lupin declarou, categórico.
--- É simplesmente impossível.

--- Mas então o que acontece com o lema dos detetives, \emph{Cherchez la
Femme}, é isso que eu gostaria de saber --- Lecoq praticamente gritou.
--- Trabalho com base nessa ideia há anos e...

--- Deixa para lá, Lecoq, meu velho --- disse Rouletabille, que era o
membro mais jovem da Sociedade --- Essa base está para lá de gasta. Eu
digo: ``Melhor fogo contra fogo e \emph{femme} contra \emph{femme}''.
Que tal?

--- Nada mal --- disse Vidocq. --- Mas quem?

%\protect\hypertarget{__DdeLink__10618_1096149864}{}{}


Kitty Ketcham?

--- Não! Ela não sabe nada chercherar. Mas conheço uma moça ---
Rouletabille disse com um ar de esperto, --- que vai resolver isso para
nós. O nome dela é Fluffy Raffles.

--- O nome basta --- Holmes disse bruscamente. --- Telefone-a agora
mesmo.

Rouletabille telefonou e, após o menor intervalo possível, uma visão
raiou na porta da Sociedade.

Era bonita, Fluffy Raffles, ah, \emph{como} era! Olhos da cor de merinó
azul-claro, bochechas como almofadinhas de cetim rosa e cabelo como uma
barra de ouro. Agora você sabe exatamente como ela era.

Fluffy vestia um vestido diáfano de chiffon fofo fulgurante e um chapéu
de jardineiro dois números maiores do que sua cabeça, com pencas e
cachos de botõezinhos de rosa ao redor. Esse era seu uniforme de
negócios, então imagem só como ela era quando se enfeitava!

Ela aceitou uma das dezessete cadeiras que os homens a ofereceram
(alguns ficaram tão distraídos que ofereceram duas ao mesmo tempo) e,
cruzando seus sapatinhos brancos (que, ainda assim, eram grandes demais
para ela!) disse, toda recatada (sua boquinha almofadada era do tipo que
sempre fala recatadamente):

--- E então?

Assim que se desatordoaram dos efeitos daquela voz de sundae de morango,
os detetives explicaram o caso para ela.

--- Deixa eu ver a fotografia --- ela disse, cheia de doçura.

Todos agarraram a foto ao mesmo tempo e ela remontou os pedaços num
instante, como se fosse um quebra-cabeças.

--- Essa é Gracie Golightly? --- ela exclamou. --- Ora, eu vi ela uma
vez, quando era só um pedacinho de gente... mas ela estava bem melhor
que isso.

--- De fato, ela decaiu um pouco --- Holmes disse, estudando o retrato.

--- Ela decaiu até o fundo do poço --- Fluffy Raffles disse, decidida.
--- Não importa. O que vocês querem que eu faça?

--- \emph{Cherchez la femme} --- Lecoq exclamou, contente em recorrer
mais uma vez à fórmula.

--- Isso! --- Fluffy exclamou com seu sorriso perolado. --- Deixa ver...
é...

Fluffy formou uma covinha no cotovelo e esticou seu belo pescocinho e
retorceu seu rosto elástico e completou todas as manobras necessárias
para enxergar seu relógio de pulso com o lado certo para cima até
anunciar a hora, triunfante, vinte e sete minutos atrasada. Mas ninguém
a corrigiu; em vez disso, todos discretamente puxaram seus ponteiros
vinte e sete minutos para trás.

--- Você vai conseguir encontrá-la? --- a Máquina de Pensar indagou,
entrelaçando e desentrelaçando seus dedos.

--- Clarquessim --- disse Fluffy, que emendava suas falas. --- Mas num
preciso sair agora. Vou me acomodar.

Atirando seu chapéu de ciranda cirandinha, ela se acomodou
confortavelmente ao telefone e pediu que lhe trouxessem chá.

Enquanto todos os detetives saíram para chercher o chá, Fluffy abriu a
lista telefônica, enorme e pesada como era, e começou a ligar para um
número depois do outro, o mais rápido que podia. Eram todos os números
de relojarias, consertadores de relógios ou simples joalherias.

Ela parou para o chá entre \emph{L} e \emph{M} e pediu um copo d'água
entre \emph{V} e \emph{W}, mas após algum tempo chamou todos os
presentes.

--- Bolas! --- ela exclamou, fingindo estar braba. --- Eu devia era ter
começado de trás para diante!

Pois foi em \emph{Zykowski} que ela acertou o lugar que queria!

Ainda assim, ela acertara, e com um sorriso recatado, sorriu para Holmes
e disse:

--- A Gracie está no número 487 da Rua 34 Norte.

--- O eterno feminino --- Holmes declarou gravemente, --- é simplesmente
uma capacidade infinita para a descoberta.

--- Que maravilha, Holmes! Que maravilha! --- exclamou o substituto do
Dr. Watson, com tanta untuosidade que recebeu um aumento ao final da
semana.

Holmes destacou todos os outros detetives para buscarem a \emph{femme}
recém-chercherada, mas eles se recusaram.

--- Telefone.

--- Avise o sobrinho, deixe que ele vá.

--- Mande um telegrama.

--- Vá você mesmo.

Essas e outras réplicas insatisfatórias responderam à ordem de Holmes.

Fluffy Raffles riu e disse:

--- Deixaqueuvô.

E então todos disseram que iriam também.

Assim eles foram buscar Gracie Golightly, a devolveram aos parentes
cheios de expectativa e então foram jantar em um salão repleto de luzes
cintilantes.

--- Conte-nos como conseguiu --- Holmes disse, saturbenigno.

--- Bem --- Fluffy começou, ampliando seu sorriso com mais um centímetro
de vermelho, --- eu percebi que Gracie não tem uma beleza fatal. Nada de
errado nisso, né, a sorte dela está nos pés, não na cara. Ela pode ser
incrível pra a mãe e pros parentes e pros copeiros, mas é feiinha de
parar relógios. Então simplesmente telefonei para ver quem havia ligado
para um relojoeiro de emergência para consertar vários relógios parados,
e descobri que o Zykowski tinha sido chamado exatamente para isso. Eu só
perguntei quem tinha feito o chamado e... feito!

--- Que maravilha, Holmes! Que maravilha! --- o Dr. Watson exclamou, de
volta ao seu lugar.

--- E foi isso que parou o relógio do jovem Ensign --- Rouletabille
refletiu. --- Ele vira a tia pela primeira vez em anos na noite anterior
e seu relógio parou no mesmo instante.

--- Sim --- Fluffy disse, criando uma covinha na face esquerda. --- Só
aquela foto da Gracie atrasou meu relógio de pulso vinte e sete minutos.
Eu devia ter usado o de tornozelo!

Fontes

The Adventure of the Mona Lisa: \emph{The Century Magazine,} Janeiro de
1912.

Sure Way to Catch Every Criminal. Ha! Ha!: \emph{The Tennessean}
(Nashville, Tennessee), 10 de março de 1912, Página 43.

The Adventure of the Lost Baby: \emph{New-York Tribune}, 23 de fevereiro
de 1913, Páginas 38-39

The Adventure of the Clothes-Line. \emph{The Century Magazine,} Maio de
1915.

Cherchez la Femme. \emph{The Green Book Magazine}, Fevereiro de 1917
